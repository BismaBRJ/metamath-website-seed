
\documentclass{slides}

%page size
\usepackage{anysize}
\papersize{8in}{11in}
%left,right,top,bottom
%\marginsize{0.4706in}{0.4706in}{0.476in}{0.4706in}
\marginsize{1in}{1in}{0.2in}{1in}

\usepackage{color}
\definecolor{mint}{rgb}{.933,1,.98}
\definecolor{darkgreen}{rgb}{0,.7,0}
\definecolor{lightgray}{gray}{0.75}
\definecolor{mscolor}{rgb}{1.0,0,0}

\usepackage{amssymb}
\usepackage{latexsym}
\usepackage{amsthm}

\begin{document}
\raggedright
\pagecolor{mint}


\begin{slide}
\begin{center}
\textcolor{blue}{\textbf{\LARGE Hilbert Lattice
Equations}}\\
\vspace{3ex}
{\large Norman Megill}\\
\vspace{1ex}
\textcolor{darkgreen}{\texttt{nm {}@ alum.mit.edu\qquad http://metamath.org}}\\
\vspace{1ex}
August 5, 2004; corrected December 20, 2004
\end{center}
\end{slide}

\begin{slide}

\begin{center}
\textcolor{blue}{\textbf{OL, OML, modular OL}}
\end{center}

An \textcolor{red}{\textit{ortholattice} (OL)} is an algebra
$\langle A,\vee,\wedge,'>$ in which the following conditions hold:
\begin{eqnarray}
a\vee b & = & b\vee a\\
(a\vee b)\vee c & = & a\vee (b\vee c)\\
a'' & = & a\\
a\vee (a\wedge b) & = & a\\
a\wedge b & = & (a'\vee b')'
\end{eqnarray}
An \textcolor{red}{\textit{orthomodular lattice} (OML)} is an OL in which
the orthomodular law holds:
\begin{eqnarray}
a\vee b & = & ((a\vee b)\wedge b')\vee b
\end{eqnarray}
A \textcolor{red}{\textit{modular} OL} is an OL in which the modular law holds:
\begin{eqnarray}
 a\vee (b\wedge (a\vee c))  & = &  (a\vee b)\wedge (a\vee c)
\end{eqnarray}


\end{slide}
\begin{slide}

\begin{center}
\textcolor{blue}{\textbf{${\cal C}({\cal H})$}}
\end{center}

A \textcolor{red}{\textit{Hilbert space}} is a (for us, complex) vector space
with an inner product, which is complete in the induced metric.

(See http:/us.metamath.org/mpegif/mmhil.html for the complete list of
axioms.)

The set of closed subspaces of a (possibly infinite-dimensional) Hilbert
space ${\cal H}$ is denoted \textcolor{red}{${\cal C}({\cal H})$}.  It
is a lattice; in particular, it is an orthomodular lattice (OML).  This
fact provides a primary motivation for studying the properties of OMLs.

${\cal C}({\cal H})$ is also called a \textcolor{red}{\textit{Hilbert lattice}}.

\end{slide}
\begin{slide}

\begin{center}
\textcolor{blue}{\textbf{${\cal C}({\cal H})$ lattice operations}}
\end{center}

The orthocomplement $a'$ of a closed subspace $a$ (or actually any
$a\subseteq{\cal H}$) is the set of vectors orthogonal
to all vectors in $a$:
\begin{eqnarray}
\textcolor{red}{a'} &
{\buildrel\rm def\over =}& \{x \in {\cal H} : (\forall y \in a)\ (x,y) = 0\}
\end{eqnarray}
where $(\cdot,\cdot)$ is the Hilbert vector space inner product.
Note that $a'$ is a closed subspace for any $a\subseteq{\cal H}$,
and $a''$ (the closure of $a$) is the smallest closed subspace
containing $a$.

The meet operation is
just set-theoretical intersection:
\begin{eqnarray}
\textcolor{red}{a\wedge b} &
{\buildrel\rm def\over =}& a\cap b
\end{eqnarray}

\end{slide}
\begin{slide}

\begin{center}
\textcolor{blue}{\textbf{${\cal C}({\cal H})$ lattice operations (cont.)}}
\end{center}

Ordering, join, unit, and zero can be defined in terms of these.  (We
also define {\em commutes} and {\em Sasaki implication} for later use.)
\begin{eqnarray}
  \textcolor{red}{a\le b} & {\buildrel\rm def\over \Leftrightarrow}&
    a=a\wedge b \quad\Leftrightarrow\quad a\subseteq b
\\
  \textcolor{red}{a\vee b} &
  {\buildrel\rm def\over =}& (a'\wedge b')' = (a{\bf +}b)''= (a\cup b)''
\\
  \textcolor{red}{0} &
  {\buildrel\rm def\over =}& a \wedge a' = \{ {\bf 0} \} = {\cal H}'
\\
  \textcolor{red}{1} &
  {\buildrel\rm def\over =}& 0' ={\cal H}
\\
  \textcolor{red}{aCb} & {\buildrel\rm def\over \Leftrightarrow}&
  a=(a\wedge b)\vee (a\wedge b') \quad\textcolor{red}{\mbox{(commutes)}}
\\
  \textcolor{red}{a\to_1b} &
  {\buildrel\rm def\over =}& a'\vee(a\wedge b)
  \ \ \quad\textcolor{red}{\mbox{(Sasaki implication)}}
\end{eqnarray}
where ${\bf +}$ is subspace sum, $\cup$ is set-theoretical union, and ${\bf 0}$ is the
zero vector.  Note that
${\cal C}({\cal H})$ itself can be defined as
\begin{eqnarray}
\textcolor{red}{{\cal C}({\cal H})}  &
{\buildrel\rm def\over =}&\{x \subseteq {\cal H} : x=x''\}
\end{eqnarray}

\end{slide}
\begin{slide}

\begin{center}
\textcolor{blue}{\textbf{An open problem in ${\cal C}({\cal H})$}}
\end{center}

``An open problem is to determine all equations satisfied by hilbertian
lattices [i.e.  ${\cal C}({\cal H})$], which would make possible the
separation of the 'purely logic' part from the above axiomatics.  It is
not known if this problem is solvable, for it is not certain that these
equations form a recursively enumerable set.''

---Ren\'e Mayet, ``Varieties of orthomodular lattices related to states,''
{\em Algebra Universalis 20} (1985), 368-396

\end{slide}
\begin{slide}

\begin{center}
\textcolor{blue}{\textbf{Equations known to hold in ${\cal C}({\cal H})$}} \\
(See also last year's slide show)
\end{center}

\begin{itemize}
\item Orthomodular (OML) law (Husumi, 1937)
\item Orthoarguesian (OA) law (Alan Day, 1975)
\item Godowski's state-related equations (Godowski, 1981)
\item Mayet's state-related equations (Mayet, 1985)
\item $n$-orthoarguesian ($n$-OA) laws (Megill/Pavi\v ci\'c, 2000)
\item The modular law {\bf does not hold} in ($\infty$-dimensional)
${\cal C}({\cal H})$
\end{itemize}


\end{slide}
\begin{slide}

\begin{center}
\textcolor{blue}{\textbf{Equations known to hold in ${\cal C}({\cal H})$ (cont.)}} \\
\end{center}

OML law (Husumi, 1937):
\begin{eqnarray}
((a\vee b)\wedge b')\vee b &=& a\vee b
\end{eqnarray}
OA law (Alan Day, 1975):
\begin{eqnarray}
 & a\wedge
   (( (a\wedge b) \vee ((a\to_1 d)\wedge (b\to_1 d)) )\vee     \nonumber\\
 & (( (a\wedge c) \vee ((a\to_1 d)\wedge (c\to_1 d)) )\wedge   \nonumber\\
 &  ( (b\wedge c) \vee ((b\to_1 d)\wedge (c\to_1 d)) )))\ \le\ b'\to_1 d
\end{eqnarray}

\end{slide}
\begin{slide}

\begin{center}
\textcolor{blue}{\textbf{Equations known to hold in ${\cal C}({\cal H})$ (cont.)}} \\
\end{center}

$n$-OA example, $n=5$ (Megill/Pavi\v ci\'c, 2000):
\begin{eqnarray}
& a\wedge (
  ((  (a\wedge b) \vee ((a\to_1 d)\wedge (b\to_1 d)) )\vee        \nonumber\\
& ((  (a\wedge e) \vee ((a\to_1 d)\wedge (e\to_1 d)) ) \wedge     \nonumber\\
& (   (b\wedge e) \vee ((b\to_1 d)\wedge (e\to_1 d)) ))) \vee     \nonumber\\
& ((( (a\wedge c) \vee ((a\to_1 d)\wedge (c\to_1 d)) )\vee        \nonumber\\
& ((  (a\wedge e) \vee ((a\to_1 d)\wedge (e\to_1 d)) ) \wedge     \nonumber\\
& (   (c\wedge e) \vee ((c\to_1 d)\wedge (e\to_1 d)) )))  \wedge  \nonumber\\
& ((  (b\wedge c) \vee ((b\to_1 d)\wedge (c\to_1 d)) ) \vee       \nonumber\\
& ((  (b\wedge e) \vee ((b\to_1 d)\wedge (e\to_1 d)) ) \wedge     \nonumber\\
& (   (c\wedge e) \vee ((c\to_1 d)\wedge (e\to_1 d)) )))
  ))\ \le\ b'\to_1 d
\end{eqnarray}

\end{slide}
\begin{slide}

\begin{center}
\textcolor{blue}{\textbf{Equations known to hold in ${\cal C}({\cal H})$ (cont.)}} \\
\end{center}

Example of Godowski equation (Godowski, 1981):
\begin{eqnarray}
(a\to_1 b)\wedge (b\to_1 c)\wedge (c\to_1 a)
&\le& a\to_1 c
\end{eqnarray}
Example of Mayet equation (Megill/Pavi\v ci\'c, unpublished):
\begin{eqnarray}
( ( a \to_1 b ) \to_1 ( c \to_1 b ) ) \wedge
                     ( a \to_1 c ) \wedge ( b \to_1 a )& \le&  c \to_1 a \ \
\end{eqnarray}

\end{slide}
\begin{slide}

\begin{picture}(0,0)(50,150)
\multiput(0,275)(0,20){6}{
  \multiput(0,0)(20,0){37}{
     \textcolor{lightgray}{\line(1,1){20}}}}
\multiput(0,275)(10,0){75}{\textcolor{lightgray}{\line(1,0){5}}}

  \put(270,360){\framebox(100,35){Boolean}}
\put(320,325){\vector(0,1){35}}
  \put(250,290){\framebox(140,35){Modular OL}}
\put(700,315){\makebox(0,0)[r]{forbidden region}}
\put(700,295){\makebox(0,0)[r]{(non-${\cal C}({\cal H})$)}}
%\put(0,275){\line(1,0){750}}

\put(700,255){\makebox(0,0)[r]{${\cal C}({\cal H})$ region}}
\put(495,245){\vector(-1,1){125}}
\put(410,260){\makebox(0,0)[l]{?}}
  %\put(445,245){\vector(-1,1){55}}
  %\put(195,245){\vector(1,1){55}}
     \multiput(445,245)(-3,3){18}{\line(1,0){0.5}}
     \put(400,290){\vector(-1,1){10}}
\put(230,260){\makebox(0,0)[r]{?}}
  %\put(195,245){\vector(1,1){55}}
     \multiput(195,245)(3,3){18}{\line(1,0){0.5}}
     \put(240,290){\vector(1,1){10}}
\put(145,245){\vector(1,1){125}}
  \put(120,210){\framebox(100,35){$n$-OA}}
  \put(420,210){\framebox(100,35){Mayet}}
\put(470,175){\vector(0,1){35}}
\put(170,175){\vector(0,1){35}}
  \put(120,140){\framebox(100,35){OA}}
  \put(415,140){\framebox(110,35){Godowski}}
\put(370,87){\vector(5,3){88}}
\put(320,105){\vector(0,1){185}}
\put(270,87){\vector(-5,3){88}}

\put(450,60){\makebox(0,0)[l]{\textcolor{blue}{Known equational}}}
\put(450,40){\makebox(0,0)[l]{\textcolor{blue}{varieties that include ${\cal C}({\cal H})$}}}

%\put(470,5){\vector(1,0){25}}
\put(495,5){\vector(-1,0){25}}
%\put(515,5){\makebox(0,0)[l]{means $\supsetneqq$}}
\put(515,5){\makebox(0,0)[l]{means $\subset$}}

  \put(270, 70){\framebox(100,35){OML}}
\put(320,35){\vector(0,1){35}}
  \put(270,  0){\framebox(100,35){OL}}
\end{picture}


\end{slide}
\begin{slide}

\begin{picture}(0,0)(50,150)
\multiput(0,275)(0,20){6}{
  \multiput(0,0)(20,0){37}{
     \textcolor{lightgray}{\line(1,1){20}}}}
\multiput(0,275)(10,0){75}{\textcolor{lightgray}{\line(1,0){5}}}

  \put(270,360){\framebox(100,35){Boolean}}
\put(320,325){\vector(0,1){35}}
  \put(250,290){\framebox(140,35){Modular OL}}
\put(700,315){\makebox(0,0)[r]{forbidden region}}
\put(700,295){\makebox(0,0)[r]{(non-${\cal C}({\cal H})$)}}
%\put(0,275){\line(1,0){750}}

\put(700,255){\makebox(0,0)[r]{${\cal C}({\cal H})$ region}}
\put(495,245){\vector(-1,1){125}}
\put(410,260){\makebox(0,0)[l]{?}}
  %\put(445,245){\vector(-1,1){55}}
  %\put(195,245){\vector(1,1){55}}
     \multiput(445,245)(-3,3){18}{\line(1,0){0.5}}
     \put(400,290){\vector(-1,1){10}}
\put(230,260){\makebox(0,0)[r]{?}}

% extra stuff
\put(240,267){\makebox(0,0)[l]{\textcolor{mscolor}{$n=3$}}}
\put(240,247){\makebox(0,0)[l]{\textcolor{mscolor}{proved}}}

  %\put(195,245){\vector(1,1){55}}
     \multiput(195,245)(3,3){18}{\line(1,0){0.5}}
     \put(240,290){\vector(1,1){10}}
\put(145,245){\vector(1,1){125}}
  \put(125,210){\framebox(100,35){$n$-OA}}
  \put(420,210){\framebox(100,35){Mayet}}

%extra stuff
\put(485,210){\textcolor{mscolor}{\vector(0,-1){35}}}
\put(481,187){\textcolor{mscolor}{\line(1,1){10}}}
\put(530,207){\makebox(0,0)[l]{\textcolor{mscolor}{Megill/Pavi\v ci\'c}}}
\put(530,187){\makebox(0,0)[l]{\textcolor{mscolor}{(unpublished)}}}
\put(155,210){\textcolor{mscolor}{\vector(0,-1){35}}}
\put(151,187){\textcolor{mscolor}{\line(1,1){10}}}
\put(123,207){\makebox(0,0)[r]{\textcolor{mscolor}{Megill/Pavi\v ci\'c}}}
\put(123,187){\makebox(0,0)[r]{\textcolor{mscolor}{(2000)}}}

\put(470,175){\vector(0,1){35}}
\put(170,175){\vector(0,1){35}}
  \put(125,140){\framebox(100,35){OA}}
  \put(415,140){\framebox(110,35){Godowski}}
\put(370,87){\vector(5,3){88}}
\put(320,105){\vector(0,1){185}}
\put(270,87){\vector(-5,3){88}}

\put(450,60){\makebox(0,0)[l]{\textcolor{blue}{Known equational}}}
\put(450,40){\makebox(0,0)[l]{\textcolor{blue}{varieties that include ${\cal C}({\cal H})$}}}

%\put(470,5){\vector(1,0){25}}
\put(495,5){\vector(-1,0){25}}
%\put(515,5){\makebox(0,0)[l]{means $\supsetneqq$}}
\put(515,5){\makebox(0,0)[l]{means $\subseteq$}}

  \put(270, 70){\framebox(100,35){OML}}
\put(320,35){\vector(0,1){35}}
  \put(270,  0){\framebox(100,35){OL}}
\end{picture}


\end{slide}
\begin{slide}

\begin{center}
\textcolor{blue}{\textbf{Modular pairs}}
\end{center}

The \textcolor{red}{\textit{modular pair}} relation between
two lattice elements $a$, $b$, denoted
 $(a,b)M$, is defined as
\begin{eqnarray}
\textcolor{red}{(a,b)M} & {\buildrel\rm def\over \Leftrightarrow}&
  (\forall x)\ [ x \le b \ \Rightarrow\  x \vee (a \wedge b) = (x \vee a) \wedge b]
 \label{mpdef}
\end{eqnarray}

The \textcolor{red}{\textit{dual modular pair}} relation between
two lattice elements $a$, $b$, denoted
 $(a,b)M^*$, is defined as
\begin{eqnarray}
\textcolor{red}{(a,b)M^*} & {\buildrel\rm def\over \Leftrightarrow}&
  (\forall x)\ [ x \ge b \ \Rightarrow\  x \wedge (a \vee b) = (x \wedge a) \vee b]
\label{dmpdef}
\end{eqnarray}

\end{slide}
\begin{slide}

\begin{center}
\textcolor{blue}{\textbf{M-symmetry}}
\end{center}

The set of closed subspaces of infinite-dimensional Hilbert space, ${\cal C}({\cal H})$,
has a remarkable property:  it is \textcolor{red}{\textit{M-symmetric}}.
\begin{eqnarray}
  (a,b)M \quad\Leftrightarrow\quad (b,a)M
\end{eqnarray}
Other related symmetry properties also hold:
\begin{eqnarray}
  (a,b)M^* &\quad\Leftrightarrow\quad& (b,a)M^* \quad\mbox{(\textcolor{red}{\textit{M${}^*$-symmetric}})}\\
  (a,b)M &\quad\Leftrightarrow\quad& (b',a')M \quad\mbox{(\textcolor{red}{\textit{O-symmetric}})}\\
  (a,b)M &\quad\Leftrightarrow\quad& (b,a)M^* \quad\mbox{(\textcolor{red}{\textit{cross-symmetric}})}
\end{eqnarray}

\end{slide}
\begin{slide}

\begin{picture}(0,0)(50,150)

\multiput(0,275)(0,20){6}{
  \multiput(0,0)(20,0){37}{
     \textcolor{lightgray}{\line(1,1){20}}}}
\multiput(0,275)(10,0){75}{\textcolor{lightgray}{\line(1,0){5}}}

  \put(270,360){\framebox(100,35){Boolean}}
\put(320,325){\vector(0,1){35}}
  \put(250,290){\framebox(140,35){Modular OL}}
\put(700,315){\makebox(0,0)[r]{forbidden region}}
\put(700,295){\makebox(0,0)[r]{(non-${\cal C}({\cal H})$)}}
  \put(220,235){\framebox(200,35){\textcolor{mscolor}{M-symmetric OML}}}
%\put(0,280){\line(1,0){750}}
\put(700,255){\makebox(0,0)[r]{${\cal C}({\cal H})$ region}}
\put(495,235){\vector(-1,1){125}}
  \put(400,210){\makebox(0,0)[l]{?}}
  %\put(220,220){\vector(2,1){30}}
     \multiput(210,215)(4,2){9}{\line(1,0){0.5}}
     \put(240,230){\vector(2,1){10}}
  \put(240,210){\makebox(0,0)[r]{?}}
  %\put(420,220){\vector(-2,1){30}}
     \multiput(430,215)(-4,2){9}{\line(-1,0){0.5}}
     \put(400,230){\vector(-2,1){10}}
\put(145,235){\vector(1,1){125}}
  \put(110,200){\framebox(100,35){$n$-OA}}
  \put(430,200){\framebox(100,35){Mayet}}
\put(480,175){\vector(0,1){25}}
\put(160,175){\vector(0,1){25}}
  \put(110,140){\framebox(100,35){OA}}
  \put(425,140){\framebox(110,35){Godowski}}
\put(370,87){\vector(5,3){88}}
\put(320,270){\vector(0,1){20}}
\put(320,105){\vector(0,1){130}}
%\put(320,105){\vector(0,1){185}}
\put(270,87){\vector(-5,3){88}}
\put(420,50){\makebox(0,0)[l]{\textcolor{mscolor}{Note: M-symmetric OML is}}}
\put(420,30){\makebox(0,0)[l]{\textcolor{mscolor}{(apparently) not an equational}}}
\put(420,10){\makebox(0,0)[l]{\textcolor{mscolor}{variety (but I have no proof)}}}
  \put(270, 70){\framebox(100,35){OML}}
\put(320,35){\vector(0,1){35}}
  \put(270,  0){\framebox(100,35){OL}}
\end{picture}


\end{slide}
\begin{slide}

\begin{center}
\textcolor{blue}{\textbf{Can M-symmetry help us find a
new ${\cal C}({\cal H})$ equation?}}
\end{center}

The M- (or M${}^*$-)symmetry property is not an equation, because
in prenex normal form it has
an existential quantifier.  To obtain an equation, one possible approach
(that we are investigating)
is to find a quantifier-free expression (polynomial equations
connected with `and') $E(a,b,\ldots )$ s.t.
\begin{eqnarray}
E(a,b,\ldots )  & \Rightarrow & (b,a)M^* \label{mshyp}
\end{eqnarray}
holds in OML (or in some other known ${\cal C}({\cal H})$ condition).  Then
\begin{eqnarray}
E(a,b,\ldots )  & \Rightarrow & (a,b)M^* \label{msconcl}
\end{eqnarray}
will also hold in ${\cal C}({\cal H})$ and (after removal of $(a,b)M^*$
quantifier) will
be an equational inference that holds in ${\cal C}({\cal H})$, hopefully
stronger than the first condition.

\end{slide}

%%%%%% start of deleted section that may be wrong - figure out later
%%%%%% deleted by ndm on 12/20/04
%
% \begin{slide}
%
% \begin{center}
% \textcolor{blue}{\textbf{Extending M${}^*$-symmetry}}
% \end{center}
%
% A detailed analysis of Maeda's M${}^*$-symmetry proof reveals that it
% actually proves something subtly ``stronger.''
%
% The M${}^*$-symmetry proof makes use of the fact that ${\cal C}({\cal
% H})$ is a {\em relatively atomic}
%  lattice satisfying the {\em exchange axiom}.
%
% So
% we need more definitions...
%
% \end{slide}
% \begin{slide}
%
% \begin{center}
% \textcolor{blue}{\textbf{More definitions}}
% \end{center}
%
%
% A lattice element $b$ \textcolor{red}{\textit{covers}} $a$, written $a\lessdot
% b$, if $a$ is less than $b$ and there is nothing in between.  An
% \textcolor{red}{\textit{atom}} is a lattice element that covers $0$.
% \begin{eqnarray}
% \textcolor{red}{a\lessdot b} & {\buildrel\rm def\over \Leftrightarrow}&
% a<b \ \&\ \lnot (\exists x)\ a<x<b \\
% \textcolor{red}{p\ \mbox{is an atom}} & {\buildrel\rm def\over \Leftrightarrow}&
% 0\lessdot p
% \end{eqnarray}
% ${\cal C}({\cal H})$ is
% \textcolor{red}{\textit{relatively atomic}}:
% \begin{eqnarray}
% a<b & \Rightarrow & (\exists\ \mbox{an atom}\ p)\
% a< a\vee p \le b
% \end{eqnarray}
% and satisfies the
% \textcolor{red}{\textit{exchange axiom}}:
% \begin{eqnarray}
% a\wedge b\lessdot b & \Leftrightarrow & a\lessdot a\vee b
% \end{eqnarray}
%
%
% \end{slide}
% \begin{slide}
%
% \begin{center}
% \textcolor{blue}{\textbf{Extending M${}^*$-symmetry (cont.)}}
% \end{center}
%
% The hypothesis of the M${}^*$-symmetry proof
%  does not require the full strength
% of $(b,a)M^*$ (Eq.~\ref{dmpdef}) to get $(a,b)M^*$,
%  but only the following ``weaker''
% condition:
% \begin{eqnarray}
%   (\forall \ \mbox{atoms}\ p)\
%   [ p \ge a \ \Rightarrow\  p \wedge (b \vee a) = (p \wedge b) \vee a]
% \end{eqnarray}
% Thus we can exploit special properties of atoms in an attempt to prove
% the condition of Eq.~\ref{mshyp}.  For example, using
% \begin{eqnarray}
% p\nleq a\vee b &  \Rightarrow & (a\vee p)\wedge (a\vee b) = a
% \end{eqnarray}
% we can weaken  Eq.~\ref{mshyp} (that we want to prove)
% into the following form, where $p$ is (optionally)
% an atom and cannot
% occur in $E(a,b,\ldots )$:
% \begin{eqnarray}
% E(a,b,\ldots ) \ \&\ a\le p\le a\vee b \ \Rightarrow \ p\le (p\wedge b)\vee a
% \end{eqnarray}
%
%
%
%
% \end{slide}
% \begin{slide}
%
% \begin{center}
% \textcolor{blue}{\textbf{Extending M${}^*$-symmetry (cont.)}}
% \end{center}
%
% Among other properties available to exploit, there is an
% extensionality-like law for eliminating atoms:
% \begin{eqnarray}
%   & (\forall \ \mbox{atoms}\ p)\
%   (p \le a \ \Rightarrow\ p\le b) \quad \Leftrightarrow \quad a\le b&
% \end{eqnarray}
% and a closure law for modular pairs that allows working directly
% in Hilbert space with vectors and subspace sums:
% \begin{eqnarray}
%     &(a,b)M \quad \Leftrightarrow \quad a{\bf +}b=a\vee b &
% \end{eqnarray}
% Also, an atom $p$ forms a modular pair with any other lattice element:
% \begin{eqnarray}
% (a,p)M,\ (p,a)M,\ (a,p)M^*, \ \mbox{etc.}
% \end{eqnarray}
%
% \end{slide}
% \begin{slide}
%
% \begin{center}
% \textcolor{blue}{\textbf{Our current working conjecture}}
% \end{center}
%
% We (myself and Mladen Pavi\v ci\'c) are trying to prove or disprove that
% this condition holds in all OMLs:
% \begin{eqnarray}
% aCc \ \& \ c\wedge d\le a \ \& \ b\le d \ \& \ b\le (a\wedge d)\vee c \ \&
% \nonumber\\
% \ p\le a\vee b \ \& \ a\le p \ \Rightarrow \ p\le (p\wedge b)\vee a \label{conj1}
% \end{eqnarray}
% So far we haven't found a counterexample.
%
% On the other hand, the conclusion using M${}^*$-symmetry,
% \begin{eqnarray}
% aCc \ \& \ c\wedge d\le a \ \& \ b\le d \ \& \ b\le (a\wedge d)\vee c \ \&
% \nonumber\\
% \& \ b\le e \ \Rightarrow \ e\wedge(a\vee b)\le (e\wedge a)\vee b,
% \end{eqnarray}
% fails in almost all non-modular OMLs and appears much stronger than, and
% certainly independent from, any known Hilbert lattice equation.
%
% \end{slide}
% \begin{slide}
%
% \begin{center}
% \textcolor{blue}{\textbf{Our current working conjecture (cont.)}}
% \end{center}
%
% \textcolor{blue}{\textbf{Motivation.}}
% The following informal heuristic motivates our conjecture.
% Trivially, we have
% \begin{eqnarray}
% a=b  & \Rightarrow & (b,a)M^*
% \end{eqnarray}
% The general idea is to disconnect $a=b$, then reconnect $a$ and $b$ via a
% modular or modular-like law.
% \begin{eqnarray}
% a=\mbox{(l.h.s. of modular law)} \ \&\ \mbox{(r.h.s. of modular law)}=b
% \nonumber\\
% \Rightarrow \quad (b,a)M^* \label{conj2}
% \end{eqnarray}
% If we select the direction and form of the modular law ``just right,'' the
% hope is that the strength of the
%  hypotheses will effectively ``cancel'' the strength of
% $(b,a)M^*$, but won't ``cancel'' the strength of $(a,b)M^*$.  Very
% roughly, we observe this kind of bias with simple lattice experiments.
% By trial-and-error we added the additional hypothesis $aCc$ in
% Eq.~\ref{conj1} to get Eq.~\ref{conj2} to pass in known OMLs.
%
% \end{slide}
% \begin{slide}
%
% \begin{center}
% \textcolor{blue}{\textbf{Our current working conjecture (cont.)}}
% \end{center}
%
%
% Eq.~\ref{conj1} is equivalent to the following closed form:
% \begin{eqnarray}
% p\wedge (a\vee ((((a\wedge d)\vee (c\wedge (a\vee (c\wedge a'))))\wedge d)\wedge b))\nonumber\\
% \le ((((c\wedge d)\vee a)\vee p)\wedge b)\vee ((c\wedge d)\vee a)
% \end{eqnarray}
%
%
% Expressed as a Mace4 problem:
% \begin{verbatim}
% % Use as the denial in the Mace4 example "nonmodular-oml.in"
% (E ^ (A v ((((A ^ D) v (C ^ (A v (C ^ c(A))))) ^ D) ^ B)))
% v (((((C ^ D) v A) v E) ^ B) v ((C ^ D) v A))
% != ((((C ^ D) v A) v E) ^ B) v ((C ^ D) v A).
% \end{verbatim}
%
% \end{slide}
%%%%%% end of deleted section that may be wrong

%%%%%%% temporary new slide:
\begin{slide}

\begin{center}
\textcolor{blue}{\textbf{Our current working conjecture}}
\end{center}


We are trying to prove or disprove that the following equation holds
in all OMLs:
\begin{eqnarray}
p\wedge (a\vee ((((a\wedge d)\vee (c\wedge (a\vee (c\wedge a'))))\wedge d)\wedge b))\nonumber\\
\le ((((c\wedge d)\vee a)\vee p)\wedge b)\vee ((c\wedge d)\vee a)
\end{eqnarray}


Expressed as a Mace4 problem:
\begin{verbatim}
% Use as the denial in the Mace4 example "nonmodular-oml.in"
(E ^ (A v ((((A ^ D) v (C ^ (A v (C ^ c(A))))) ^ D) ^ B)))
v (((((C ^ D) v A) v E) ^ B) v ((C ^ D) v A))
!= ((((C ^ D) v A) v E) ^ B) v ((C ^ D) v A).
\end{verbatim}

\end{slide}


\begin{slide}

\begin{center}
\textcolor{blue}{\textbf{References}}

Most of the references for this material can be found at:

\textcolor{darkgreen}{
\texttt{http://us.metamath.org/qlegif/mmql.html{\char`\#}ref}}

\vspace{3ex}

More miscellaneous stuff can be found at:\\
\textcolor{darkgreen}{
\texttt{http://us.metamath.org/award2003.html}}

\end{center}
\end{slide}

\end{document}


